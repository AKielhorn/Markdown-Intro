% !TEX TS-program = pdflatex
% !TEX encoding = UTF-8 Unicode

\documentclass[11pt,ngerman,a4paper]{article}
\usepackage[T1]{fontenc}

\usepackage{ifxetex,ifluatex}
\usepackage{fixltx2e} % provides \textsubscript
% use microtype if available
\IfFileExists{microtype.sty}{\usepackage{microtype}}{}
\ifxetex
  \usepackage{fontspec,xltxtra,xunicode}
  \defaultfontfeatures{Mapping=tex-text,Scale=MatchLowercase}
  \newcommand{\euro}{€}
  \usepackage{polyglossia} % -ak-
  \setmainlanguage{german} % -ak-
\else
  \ifluatex
    \usepackage{fontspec}
    \defaultfontfeatures{Mapping=tex-text,Scale=MatchLowercase}
    \usepackage[ngerman]{babel} % -ak-
    \newcommand{\euro}{€}
  \else
    \usepackage[utf8]{inputenc}
    \usepackage[TS1,T1]{fontenc} % -ak- T1 für \textdblquotedown
%   \usepackage{lmodern} % -ak-
    \usepackage[ngerman]{babel} % -ak-
    \usepackage{eurosym}
  \fi
\fi
\usepackage[margin=1in]{geometry}
\usepackage{biblatex}
\bibliography{Ziele-md}
\usepackage{xcolor}
\usepackage{listings}
\lstdefinestyle{DTKlstBase}{%
     basicstyle=\small\ttfamily,
     basewidth=0.55em,
     keepspaces,
     identifierstyle=\color{black},
     keywordstyle=\color{black},
     stringstyle=\itshape\color{black},
     commentstyle=\itshape\color{black},
     columns=fullflexible,
     tabsize=2,
     frame=single,
     extendedchars=true,
     showspaces=false,
     showstringspaces=false,
     breaklines=true,
     breakindent=0.5em,
     backgroundcolor=\color{black!5},
     breakautoindent=true,
     captionpos=t,
     aboveskip=\medskipamount,
     belowskip=\medskipamount,
     xrightmargin=\fboxsep,
     emphstyle=\DTK@lst@colorbox{lightgray},
     prebreak=,
     postbreak=\mbox{\ensuremath{\hookrightarrow}},
     literate={ä}{{\"a}}1
              {ö}{{\"o}}1 
              {ü}{{\"u}}1
              {Ä}{{\"A}}1
              {Ö}{{\"O}}1
              {Ü}{{\"U}}1
              {ß}{\ss}1
              {”}{''}1
              {“}{\grqq}1
              {„}{\glqq}1
}

\lstdefinestyle{DTKlstNoNumber}{%
     style=DTKlstBase,
     numbers=none,
     xleftmargin=\fboxsep,
}

\let\verbatim=\relax
\lstnewenvironment{verbatim}
{\lstset{language={},style=DTKlstNoNumber}}
{}

\ifxetex
  \usepackage[setpagesize=false, % page size defined by xetex
              unicode=false, % unicode breaks when used with xetex
              xetex]{hyperref}
\else
  \usepackage[unicode=true]{hyperref}
\fi
\hypersetup{breaklinks=true,
            bookmarks=true,
            pdfauthor={Axel Kielhorn},
            pdftitle={Viele Ziele Veröffentlichung},
            colorlinks=true,
            urlcolor=blue,
            linkcolor=magenta,
            pdfborder={0 0 0}}
\urlstyle{same}  % don't use monospace font for urls
\setlength{\parindent}{0pt}
\setlength{\parskip}{6pt plus 2pt minus 1pt}
\setlength{\emergencystretch}{1em}  % prevent overfull lines
\setcounter{secnumdepth}{0}

\title{Viele Ziele Veröffentlichung}
\author{Axel Kielhorn}
\date{2013-02-03}
\usepackage{csquotes}

\begin{document}
\maketitle

\section{Anmerkung}

Dieser Artikel wurde ursprünglich für die dante Vereinszeitschrift
\enquote{Die TeXnische Komödie} 3/2011 in LaTeX geschrieben. Eine
englische Übersetzung erschien im \enquote{TUGboat} Volume 32 (2011),
No. 3.

Mit dem Erscheinen von pandoc 1.9 wurde er nach \texttt{markdown}
konvertiert. Durch eine angepasste \texttt{template} Datei wird das
Aussehen des Originalartikels nachgebildet.

\section{Ein Weg führt zu einem Ziel}

Bisher war das Ziel meiner Veröffentlichungen immer Papier: DIN~A4,
DIN~A5 oder auch mal 9~cm $\times$ 12~cm. Auf dem Weg dahin entstand
immer eine PDF-Datei, daher lag es nahe, diese am Bildschirm zu lesen,
was zumindest bei DIN~A5 bequem möglich ist.

Doch dann kamen die Mobilgeräte. Einige davon können auch PDF anzeigen
und mit etwas Aufwand kann man den Text so formatieren, das er auf
\emph{einem} Mobilgerät mit wenig scrollen lesbar ist.

Günstiger wäre natürlich ein Format, bei dem der Leser die Textgröße
bestimmen kann und das Gerät den Text passend zur Anzeigengröße
umbricht. Ein solches Format ist z. B. ePUB. Im Prinzip ist das nichts
anderes als ein ZIP-Archiv mit einer definierten Struktur und ein paar
XML-Dateien, die den eigentlichen Text enthalten. Das Aussehen kann
durch eine css-Datei gesteuert werden.

\section{Eine Umleitung}

Zum Glück gibt es ein Programm, das LaTeX~lesen und ePUB schreiben kann:
pandoc\autocite{pandoc}\footnote{pandoc ist unter GPL lizensiert.}. Wenn
die LaTeX-Datei nicht zu kompliziert ist, kann pandoc sie verstehen und
konvertieren. Aber was ist zu kompliziert? Am einfachsten konvertiert
man ein Dokument von LaTeX~nach LaTeX~und sieht was übrig bleibt.

\begin{verbatim}
pandoc -f latex -t latex --template=./default-de.latex 
-o quelle-pd.tex quelle.tex
\end{verbatim}

Dieser Befehl benutzt eine an die deutsche Sprache angepasste Version
der Standardvorlage \texttt{default.latex} um eine neue LaTeX-Datei zu
erstellen.\footnote{Das Begleitmaterial zu diesem Artikel befindet sich
  auf
  \url{https://github.com/AKielhorn/Markdown-Intro/blob/master/Ziele.zip}}

\subsection{Die Steine auf dem Weg}

Pandoc arbeitet mit UTF-8 Dateien. Latin-1 Texte lassen sich leicht
konvertieren, Umlaute gemäß \texttt{german.sty} gehen jedoch verloren.
Auch TeX-Akzente \texttt{\textbackslash{}\^{}o} oder
Einbuchstabenbefehle \texttt{\textbackslash{}o} führen manchmal zu
Problemen. Das lässt sich aber mit ein paar Zeilen \texttt{sed} beheben
(siehe Abschnitt Reisevorbereitung).

Einfache Auszeichnungsbefehle für \textbf{fett} und \emph{kursiv} werden
unterstützt, doch schon beim geschachteltem
\texttt{\textbackslash{}emph} ist Schluss.

\section{Ein neuer Anfang}

Steht die LaTeX-Datei wirklich am Anfang? Oder sollten wir nicht lieber
LaTeX~als \emph{ein} Backend verstehen und die LaTeX-Datei somit nur als
ein Zwischenprodukt?

\section{Eine ungewöhnliche Richtung: Markdown statt markup}

Markdown ist eine von John Gruber\autocite{gruber} entwickelte
Auszeichnungssprache, die so aussieht, als ob sie keine Auszeichnungen
enthält:

\begin{quote}
A Markdown-formatted document should be publishable as-is, as plain
text, without looking like it's been marked up with tags or formatting
instructions.
\end{quote}

Der Anfang dieses Artikels sieht in Markdown so aus:

\begin{verbatim}
# Ein Weg führt zu einem Ziel

Bisher war das Ziel meiner Veröffentlichungen immer Papier: DIN
A4, DIN A5 oder auch mal 9 cm $\times$ 12 cm. Auf dem Weg dahin
entstand immer eine PDF-Datei, daher lag es nahe, diese am
Bildschirm zu lesen, was zumindest bei DIN A5 bequem möglich
ist.

Doch dann kamen die Mobilgeräte. Einige davon können auch PDF
anzeigen und mit etwas Aufwand kann man den Text so formatieren,
das er auf *einem* Mobilgerät mit wenig scrollen lesbar ist.
\end{verbatim}

Dieser Text wurde mit dem Befehl

\begin{verbatim}
pandoc -f latex -t markdown -o Ziele.md Ziele.tex
\end{verbatim}

direkt aus der LaTeX-Datei erzeugt.

Markdown ist eine sehr eingeschränkte Sprache, die Manpage, die den
Sprachumfang beschreibt ist nur 16 Seiten lang. Im Vergleich dazu ist
l2kurz dreimal so lang und kratzt gerade mal an der Oberfläche von
LaTeX.

Bei der Konvertierung von LaTeX~zu Markdown sollte man beachten, das
pandoc kein TeX~versteht. Es nutzt Reguläre Ausdrücke um LaTeX~zu
interpretieren. Das hat zur Folge, das bei einigen Befehlen zusätzliche
Leerzeilen erforderlich sind, damit pandoc die Befehle von normalem Text
unterscheiden kann.

\section{Auf neuem Weg zum alten Ziel: PDF aus Markdown über LaTeX}

\begin{verbatim}
pandoc -f markdown -t latex --template=./default-de.latex 
    -o quelle.tex quelle.md
\end{verbatim}

\subsection{Die \texttt{default-de.latex} Dateien}

Die mitgelieferte \texttt{default.latex} Datei ist eine Minimalversion.
Für deutsche Texte empfiehlt sich die erweiterte
\texttt{defaut-de.latex} Datei, die im
Begleitmaterial~\autocite{Ziele-git} zu finden ist. Die Änderungen zur
Originalversion sind durch ein \texttt{-ak-} gekennzeichnet.

Seit Version 1.9 liefert Pandoc eine universelle Template-Datei mit.
Diese kann durch zuweisen einer Variablem auf die gewünschte Sprache
eingestellt werden. Das Argument ist hier eine von \texttt{babel}
unterstützte Sprache.

\begin{verbatim}
pandoc -f markdown -t latex -V lang=ngerman -s 
    -o quelle.tex quelle.md
\end{verbatim}

In diesem Fall muss pandoc mit der Option \texttt{-s} (standalone)
mitgeteilt werden, das ein vollständiges Dokument erstellt werden soll.

Leider fehlt in der offiziellen \texttt{default.latex} Datei die Zeile

\begin{verbatim}
\usepackage[T1]{fontenc}
\end{verbatim}

daher ist \texttt{default-de.latex} immer noch erforderlich.

Die \texttt{T1} Zeichensatzkodierung ist notwendig, da \texttt{OT1} die
Anführungszeichen unten nicht enthält. Französische Anführungszeichen,
«Das sind diese» funktionieren zwar ohne \texttt{fontenc}, das Ergebnis
sieht aber furchtbar aus.

Mit Version 1.9.4 wurde die template Datei erweitert und enthält nun die
Zeilen

\begin{verbatim}
\usepackage[T1]{fontenc}
\usepackage{lmodern}
\end{verbatim}

Damit ist die Datei \texttt{default-de.latex} nicht mehr erforderlich,
alle Einstellungen können über Variablen vorgenommen werden.

Anführungszeichen führen noch zu einem weiteren Problem. Als TeX noch
auf eine 7-Bit Eingabekodierung beschränkt war, wurden einige Zeichen
über Ligaturen eingegeben. Dazu zählten auch die englischen
Anführungszeichen \texttt{``} und \texttt{''}. Aus
Kompatibilitätsgründen gibt Pandoc die Anführungszeichen immer noch so
aus. In deutschen Texten werden die \texttt{``} Zeichen als
\emph{schließende} Anführungszeichen verwendet. Endet nun ein Zitat mit
einem \texttt{!} oder einem \texttt{?}, so greift ein anderer
Ligaturmechanismus, der die Kombination \texttt{! `} in ein \texttt{¡}
und \texttt{? `} in ein \texttt{¿} umwandelt. Dies lässt sich umgehen,
indem man bei der Eingabe \texttt{"} benutzt. Wenn im LaTeX template das
Paket \texttt{csquotes} benutzt wird, werden die geraden
Anführungsstriche in \texttt{\textbackslash{}enquote} Befehle
übsersetzt. In Verbindung mit der Option \texttt{german} werden daraus
dann die richtigen Anführungsstriche. Bei anderen Formaten muss man
damit leben, das hier englische Anführungsstriche verwendet werden.

Alternativ kann man natürlich die LaTeX Datei mit \texttt{sed}
bearbeiten und die Zeichen nach Unicode konvertieren.

Bei der Verwendung von französischen Anführungszeichen gibt es keine
Probleme.

Seit pandoc 1.9.3 gibt es einen neuen Schalter
\texttt{-{}-no-tex-ligatures} mit dem die Übersetzung von
Unicode-Zeichen in TeX-Ligaturen abgeschaltet werden kann. \texttt{„}
und \texttt{``} werden dann unverändert in den TeX Text übernommen.
Gleichzeitig wird die Übersetzung von \texttt{-{}-} in \texttt{--} und
\texttt{-{}-{}-} in \texttt{---} abgeschaltet. Diese Zeichen müssen dann
durch TeX konvertiert werden. Bei \texttt{pdflatex} ist das
standardmäßig der Fall, bei \texttt{XeTeX} und \texttt{LuaTeX} muss die
\texttt{fontspec} Option \texttt{Ligatures=TeX} gesetzt werden.

Dieser Schalter schaltet auch die Option \texttt{-{}-smart} ab.
Schreibmaschinenanführungszeichen \texttt{"} werden dann nicht mehr in
typographische Anführungszeichen konvertiert. Möchte man diese Funktion
erhalten, so ist die Option \texttt{-{}-smart} (Kurzform \texttt{-S})
zusätzlich anzugeben.

Eine einfache Möglichkeit, die komfortable Eingabe von \texttt{"} zu
nutzen und gleichzeitig richtige Anführungszeichen im Zieldokument zu
erhalten ist:

\begin{verbatim}
pandoc --smart file.md -t markdown 
| sed -f smart2de 
| pandoc -f markdown -o file.epub
\end{verbatim}

Die Datei \texttt{smart2de} enthält die folgenden Befehle:

\begin{verbatim}
s/“/„/g
s/”/“/g
\end{verbatim}

Natürlich kann man alternativ auch \texttt{»} und \texttt{«} als
Ersetzungstext benutzen

\subsection{Abkürzung}

Der schnellste Weg aus einer Markdown-Datei ein PDF zu erstellen ist:

\begin{verbatim}
pandoc --template=./default-de.latex -o quelle.pdf quelle.md
\end{verbatim}

Am Zielformat \texttt{pdf} erkennt pandoc, das es die PDF-Datei direkt
erzeugen soll. Im Hintergrund wird natürlich LaTeX aufgerufen, es
bleiben jedoch keine temporären Dateien zurück. Wird eine
Inhaltsverzeichnis gewünscht, ist ein zweiter LaTeX-Lauf erforderlich.
Pandoc erkennt das am Befehl \texttt{\textbackslash{}tableofcontents}.

Mit den Schaltern \texttt{-{}-xetex} bzw. \texttt{-{}-luatex} kann man
von pdfTeX~ auf XeLaTeX oder LuaLaTeX umschalten. Im
\texttt{default.latex} wird mit Hilfe von \texttt{ifxetex} und
\texttt{ifluatex} ermittelt, welches TeX verwendet wird.

Auf diesem Weg kann man natürlich kein BibTeX/Biber verwenden. Zum
Erstellen des Literaturverzeichnisses muss man auf die pandoc internen
Funktionen zurückgreifen.

Bei älteren pandoc Versionen übernimmt das mitgelieferte Programm
\texttt{markdown2pdf} die Konvertierung nach PDF.

\begin{verbatim}
markdown2pdf --template=./default-de.latex quelle.md
\end{verbatim}

\subsection{Nacharbeit}

Die automatisch erstellte LaTeX-Datei ist fast so gut wie eine von einem
Anfänger manuell erstellte. Natürlich wird man auch hier mit Trennhilfen
und ähnlichem die overfull und underfull hboxes beseitigen müssen.

\section{Der Weg zu einem neuen Ziel: ePUB}

Der ursprüngliche Wunsch war es, eine ePUB Datei zu erstellen. Dies
geschieht mit dem Befehl:

\begin{verbatim}
pandoc -f markdown -t epub --epub-cover-image=cover-image.gif -s
    -o Ziele.epub Ziele.md
\end{verbatim}

Der Text wird anhand der Gliederungsbefehle in verschiedene Dateien
aufgeteilt, das erleichtert die Nachbehandlung, z. B. mit
Sigil~\autocite{sigil}.

Seit Version 1.8.1.2 kann ein Titelbild mit der Option
\texttt{-{}-epub-cover-image} eingebunden werden.

Damit ein eBook-Reader den Text richtig umbrechen und bei Bedarf trennen
kann, ist es erforderlich ihm die Dokumentsprache mitzuteilen. Pandoc
wertet die Umgebungsvariable \texttt{\$LANG} aus und setzt die Sprache
entsprechend. Auf einem deutschen Rechner hat diese Variable
normalerweise den Wert \texttt{de\_DE}, in Österreich den Wert
\texttt{de\_AT}.

Möchte man einen Text in einer anderen Sprache schreiben, so kann man
diesen Wert durch zuweisen einer Variablen ändern.

\begin{verbatim}
pandoc -f markdown -t epub  -s -V lang=en_US
    -o Ziele.epub Ziele.md
\end{verbatim}

führt zu einem Text in amerikanischem English.

Auf einigen ebook Readern kann es zu Darstellungsproblemen kommen, da
die installierten Zeichensätze nur einen kleinen Teil des Unicode
Zeichenvorrats enthalten. Feste Abstände, wie sie z. B. im
Mathematiksatz verwendet werden, erscheine dann als Fragezeichen. Oft
fehlt auch der Pfeil mit Haken, der als Zurück-Symbol in Fußnoten
verwendet wird.

Bei den meisten Geräten lassen sich eigene Zeichensätze installieren.
Gute Erfahrungen habe ich mit GNU Freefont
(\url{http://savannah.gnu.org/projects/freefont/}) und DejaVu
(\url{http://dejavu-fonts.org/wiki/index.php}) gemacht.

\subsection{Ein Lagerfeuer}

Es gibt zwei Wege, um aus ePUB Dateien Dokumente für den
Kindle\footnote{Kindle ist wahrscheinlich ein eingetragenes Warenzeichen
  von Amazon, auch wenn ich auf den ersten Blick keinen entsprechenden
  Hinweis auf der Webseite gefunden habe.} zu erstellen:

\begin{enumerate}
\def\labelenumi{\arabic{enumi}.}
\item
  Der offizielle Amazon Weg führt über das Programm \texttt{kindlegen},
  das von der Amazon Webseite heruntergeladen werden kann. Ab pandoc
  Version 1.9 kann \texttt{kindlegen} die von pandoc erzeugten Dateien
  direkt in das \texttt{mobi}-Format übersetzen.

  \texttt{Kindlegen} mag es nicht, wenn als Sprache \texttt{de\_DE}
  angegeben ist. Soll ein ePUB später mit \texttt{kindlegen}
  weiterverarbeitet werden, so ist als Sprache \texttt{de} für deutsch,
  ohne die Landesangabe \texttt{DE} oder \texttt{AT} zu wählen.
  Schweitzer Deutsch \texttt{de\_CH} ist jedoch in Ordnung, es wird
  wahrscheinlich wegen des fehlenden \texttt{ß} anders behandelt. Beim
  Aufruf von pandoc ist die Option

\begin{verbatim}
-V lang=de
\end{verbatim}

  anzugeben.

  Ältere pandoc Versionen erzeugen ein ePUB, bei dem \texttt{kindlegen}
  Probleme hat, wenn im Titel oder den Kapitelüberschriften Umlaute
  vorkommen. Man kann das Problem umgehen, indem man das Dokument in
  \texttt{sigil} öffnet und gleich wieder speichert. Alternativ kann man
  es auch mit den Komandozeilenwerkzeugen aus
  \texttt{calibre}~\autocite{calibre} von ePUB nach ePUB konvertieren.
  Danach verarbeitet \texttt{kindlegen} die Datei ohne Probleme.

\begin{verbatim}
ebook-convert quelle.epub quelle_fixed.epub 
    -no-default-epub-cover
\end{verbatim}

  Mit dem Erscheinen des Kindle fire hat Amazon ein neues Dateiformat
  \texttt{KF8} eingeführt. Zur Konvertierung in dieses Format ist
  \texttt{kindlegen} Version 2.0 erforderlich.
\item
  Eine Open Source Alternative ist das Programm
  \texttt{calibre}~\autocite{calibre}. Im Menüpunkt
  \texttt{Einstel"-lungen} kann man unter
  \texttt{Erweitert-\textgreater{} Verschiedenes-\textgreater{} Komandozeilen-Tools installieren}
  Kommandozeilenwerkzeuge installieren. Diese stehen dann z. B. für ein
  Makefile zur Verfügung. Beim Ausgabeformat \texttt{mobi} für den
  Kindle lautet die Kommandozeile:

\begin{verbatim}
ebook-convert quelle.epub quelle.mobi --output-profile=kindle 
\end{verbatim}

  Die erstellte Datei ist größer als bei \texttt{kindlegen}, da der Text
  weniger gut komprimiert wird.

  Mehr Informationen zu dem Befehl erhält man mit:

\begin{verbatim}
ebook-convert quelle.epub quelle.mobi -h 
\end{verbatim}
\end{enumerate}

\section{Sehen wo ich gehe: WYSIWYG}

\enquote{Kann ich das als Word-Datei haben?} Wer kennt diese Frage
nicht? Mit pandoc 1.9 ist es möglich eine \texttt{docx} Datei zu
erstellen:

\begin{verbatim}
pandoc -f markdown -t docx --reference-docx=./reference-de.docx -s 
    -o quelle.odt quelle.md
\end{verbatim}

Die Sprache der Datei ist englisch und muss nach jeder Konvertierung in
Formatvorlage \texttt{Standard} auf deutsch zurückgesetzt werden.

Wer lieber mit StarOffice / OpenOffice / LibreOffice arbeitet, kann mit

\begin{verbatim}
pandoc -f markdown -t odt --reference-odt=./reference-de.odt -s 
    -o quelle.odt quelle.md
\end{verbatim}

eine passende Datei erstellen.

Die Dateien \texttt{reference-de.docx} bzw. \texttt{reference-de.odt}
dienen hier als Basis für die Absatzvorlagen und das Dokumentlayout.
Dies können eine beliebige Dateien sein, am einfachsten ist es jedoch
eine von pandoc erstellte Datei an die eigenen Layoutwünsche anzupassen.
So ist sichergestellt, das die internen Bezeichnungen der Absatzformate
mit den von pandoc vergebenen übereinstimmen.

Die Datei \texttt{reference-de.odt} basiert auf einer von pandoc
erstellten Datei, bei der die Sprache auf deutsch umgestellt wurde. Bis
pandoc 1.8.1.2 gab es ein Problem beim Einbinden von Bildern. OpenOffice
meldet eine korrupte Datei und bietet an, diese zu reparieren. Dieser
Fehler wurde in Version 1.8.1.3 behoben.

Eingebundene Bilder werden auf Einheitsgröße zusammengestaucht und
müssen jeweils einzeln auf Originalgröße expandiert werden. Dieser
Fehler wurde mit pandoc 1.9 behoben.

\section{Reisevorbereitung}

Mit einem kleinen \texttt{sed}-Skript lässt sich diese
\texttt{dtk}-basierte LaTeX-Quelle in eine neutrale LaTeX-Datei
umwandeln, die von pandoc weitgehend verlustlos gelesen werden kann.
Lediglich die \texttt{bibliography}-Umgebung geht verloren.

\begin{verbatim}
s/\\LaTeX/LaTeX/g
s/\\TeX/TeX/g
s/\\ConTeXt/ConTeXt/g
s/\\Program{\([a-zA-Z]*\)}/\1/g
s/\\Acronym{\([a-zA-Z]*\)}/\1/g
s/\\Package/\\texttt/g
s/\[style=DTKlstNoNumber\]//
s/\\File/\\texttt/g
s/\\Macro{/\\texttt{\\textbackslash /g
s/\\begingroup//
s/\\endgroup//
\end{verbatim}

Mit dem Aufruf

\begin{verbatim}
sed -f dtk2mdtex.sed Ziele.tex >Ziele-clean.tex
\end{verbatim}

werden die LaTeX-Befehle, die pandoc nicht kennt, aus der Quelldatei
entfernt. Das Ergebnis kann dann mit

\begin{verbatim}
pandoc -f latex -t markdown -s -o Ziele-clean.md  Ziele-clean.tex
\end{verbatim}

nach Markdown konvertiert werden.

\section{Wegbeschreibung}

\subsection{Gliederungsbefehle}

Markdown unterstützt bis zu 6 Gliederungsebenen. Die Gliederungsstufe
wird durch die Anzahl der \texttt{\#} Zeichen vorgegeben. Vor jeder
Überschrift ist eine Leerzeile erforderlich.

\begin{verbatim}
# Oberste Ebene

## Zweite Ebene

### Dritte Ebene

#### *Wichtige Information* in der vierten Ebene versteckt
\end{verbatim}

Eine alternative Form der Gliederungsbefehle unterstützt nur zwei
Ebenen:

\begin{verbatim}
Erste Ebene
===========

Zweite und letzte Ebene
-----------------------
\end{verbatim}

Weiter Ebenen müssen dann durch \texttt{\#} Zeichen angegeben werden.
Standardmäßig erzeugt pandoc beim konvertieren nach Markdown dieses
Format. Mit der Option \texttt{-{}-atx-headers} werden Überschriften im
ersten Format erstellt.

\subsection{Zitatumgebung}

Zitiert wird wie in (alten) E-Mail Programmen mit einem
\texttt{\textgreater{}}

\begin{verbatim}
> Hier zitiere ich eine alte E-Mail
>
> > Und hier wird in der alten E-Mail eine noch ältere zitiert.
>
> Normalerweise erhält jede Zeile ein >-Zeichen. Einige Editoren
> fügen dann in der Folgezeile das > gleich mit ein.
>
> Sollte ein Editor nicht über diese Funktion verfügen,
reicht es, nur die erste Zeile eines Absatzes zu markieren.
\end{verbatim}

Der Text wird dabei normal umgebrochen. Möchte man den Zeilenumbruch der
Quelldatei im fertigen Dokument erhalten, so gibt es dafür den
Zeilenblock. Er wird durch \texttt{\textbar{}} eingeleitet. Somit lassen
sich Gedichte formatieren. Leerzeichen am Zeilenanfang bleiben erhalten.
Anders als Programmlisings, die in Schreibmaschinenschrift gesetzt
werden, wird hier die normale Textschrift benutzt.

\begin{verbatim}
| Vom Eise befreit sind Straßen und Wege,
| Durch des Salzes ätzende Kraft.
|     Und so weiter.
\end{verbatim}

Eine besondere Form von Zitaten sind Zitate aus Programmen. Diese werden
standardmäßig in einer nichtproportionalen Schrift gesetzt.
Programmzitate werden durch 4 Leerzeichen oder einen Tabulatorschritt
eingegeben:

\begin{verbatim}
    \documentclass[11,ngerman]{dtk}
    \usepackage[utf8]{inputenc}
\end{verbatim}

Auch vor Zitaten ist eine Leerzeile erforderlich.

Will man bei längeren Listings nicht jede Zeile einrücken, so gibt es in
pandoc eine alternative Form, diese ist jedoch nicht Standard-Markdown.
Durch eine Zeile mit mindestens 3 \texttt{\textasciitilde{}}-Zeichen
wird das Programmlisting eingeleitet und durch eine Zeile mit mindestens
genauso vielen \texttt{\textasciitilde{}}-Zeichen wieder beendet. Eine
Zeile mit weniger \texttt{\textasciitilde{}}-Zeichen ist Bestandteil des
Listings.

\begin{verbatim}
~~~~~~~~
Dies ist ein Programm Listing
~~~~
Header
~~~~
Body
~~~~~~~~
\end{verbatim}

Pandoc 1.9 wird standardmäßig mit Unterstützung für Syntaxhervorhebung
übersetzt. Bei älteren Version musste man das explizit aktivieren. Eine
Liste der unterstützten Sprachen erhält man mit:

\begin{verbatim}
pandoc -v
\end{verbatim}

Die Sprache teilt man dem Listing mit, außerdem ist es möglich die
Zeilen zu nummerieren:

\begin{verbatim}
~~~~~~~~{.Latex .numberLines startFrom="10"}
\documentclass[11,ngerman]{dtk}
\usepackage[utf8]{inputenc}
~~~~~~~~
\end{verbatim}

Mit dem Schalter \texttt{-{}-no-highlight} kann man die Hervorhebung
abschalten.

Mit dem Schalter \texttt{-{}-highlight-style} lassen sich verschiedene
Stile auswählen. Zur Auswahl stehen: \emph{pygments} (the default),
\emph{kate}, \emph{monochrome}, \emph{espresso}, \emph{zenburn},
\emph{haddock}, und \emph{tango}.

Die Hervorhebung funktioniert nur bei den Ausgabeformaten \texttt{HTML}
und \texttt{LaTeX}.

\subsection{Listen}

Aus LaTeX~kennen wir verschiedene Listenformen:

\subsubsection{Die \texttt{itemize} Liste}

Eine Liste wird durch ein Aufzählungszeichen (\texttt{*}, \texttt{+}
oder \texttt{-}) eingeleitet.

\begin{verbatim}
* eins
* zwei
* drei
    - drei a
    - drei b
* vier
  Wie immer verstecken wir die wichtigen Informationen ganz unten,
  in der Hoffnung, das niemand sie liest.
\end{verbatim}

Besteht ein Listeneintrag aus mehreren Absätzen, so sind die
Folgeabsätze mit 4 Leerzeichen oder einem Tabulatorschritt einzurücken.

\begin{verbatim}
* eins
* zwei
* drei
    - drei a
    - drei b
* vier

Wie immer verstecken wir die wichtigen Informationen ganz
unten, in der Hoffnung, das niemand sie liest.
    
Um ganz sicher zu gehen, erwähnen wir erst im letzten Absatz
die vier Leerzeichen Regel.
\end{verbatim}

\subsubsection{Die \texttt{enumerate} Liste}

Das erste Aufzählungszeichen gibt hier das Format der Aufzählungszeichen
vor (\texttt{1.}, \texttt{(1)} oder \texttt{i.}). Es ist nicht
notwendig, das die weiteren Aufzählungszeichen in der richtigen
Reihenfolge erscheinen (auch wenn das seltsam aussieht).

Es wird automatisch das \texttt{enumerate}-Paket geladen. Möchte man das
verhindern, so kann man stattdessen die Aufzählungspunkte mit
\texttt{\#.} markieren, dann wird die Standardeinstellung der
verwendeten Dokumentklasse benutzt.

Ab Version 1.10 wird das \texttt{enumerate}-Paket nicht mehr benötigt,
die Anpassungen erfolgen direkt im TeX-Code.

\begin{verbatim}
1. ein
2. zwei
4. drei
    a) drei a
    b) drei b
5. vier
  Wie immer verstecken wir die wichtigen Informationen ganz unten,
  in der Hoffnung, das niemand sie liest.
\end{verbatim}

Pandoc kann auch Listen mit römischen Zahlen nummerieren, dabei gibt es
jedoch einen Stolperstein: Damit nicht abgekürzte Vornamen die zufällig
am Anfang einer Zeile stehen als Aufzählzeichen für eine Liste erkannt
werden, ist bei den Zahlen I, V, X, C, D und M ein doppeltes Leerzeichen
erforderlich. Bei Kleinbuchstaben ist das nicht notwendig.

\begin{verbatim}
1. ein
2. zwei
4. drei
    I.  drei a (mit doppeltem Leerzeichen)
    II.  drei b
5. vier
    i. vier a (ohne doppeltes Leerzeichen)
    ii. vier b
\end{verbatim}

\subsubsection{Die \texttt{description} Liste}

Endlich kommen wir zu den beliebten \texttt{l2kurz}-Tieren. Das
beschriebene Objekt steht allein in einer Zeile, die Beschreibung wird
durch einen Doppelpunkt oder eine Tilde eingeleitet.

Wenn \emph{hinter} der Beschreibung eine Leerzeile steht, wird sie als
Absatz formatiert, also gegebenenfalls mit Einzug und größerem
Zeilenabstand. Um eine kompakte Liste zu erreichen, muss man auf Absätze
verzichten, bzw. diese müssen als mehrere Beschreibungen definieren
werden. Wichtig ist hierbei auch der Abstand hinter der letzten
Beschreibung, sonst wird u.U. der letzte Punkt der Liste anders
formatiert als die übrigen Punkte.

Beim OpenDocument- oder LaTeX-Export werden beide Fälle gleich
behandelt. Ab Version 1.10 wird auch beim LaTeX-Export zwischen
kompakten Listen und Listen mit mehr Freiraum unterschieden.

\begin{verbatim}
Gelse
  : ein kleines Tier, das östlich des Semmering Touristen verjagt.
Gemse
  : ein großes Tier, das westlich des Semmering von Touristen 
    verjagt wird.
  : Langer Absatz über die Frage ob es Gemsen oder Gämsen heißt.
Gürteltier
  ~ ein mittelgroßes Tier, das hier nur wegen der Länge seines 
    Namens vorkommt.
  ~ Gürteltiere sind in Österreich außerhalb von Zoologischen 
    Gärten selten anzutreffen.
Nächster Absatz
\end{verbatim}

Die l2kurz Tiere, hier mit etwas mehr Platz:

\begin{verbatim}
Gelse
  : ein kleines Tier, das östlich des Semmering Touristen verjagt.

Gemse
  : ein großes Tier, das westlich des Semmering von Touristen 
    verjagt wird.
  : Langer Absatz über die Frage ob es Gemsen oder Gämsen heißt.

Gürteltier
  ~ ein mittelgroßes Tier, das hier nur wegen der Länge seines 
    Namens vorkommt.
  ~ Gürteltiere sind in Österreich außerhalb von Zoologischen 
    Gärten selten anzutreffen.

Nächster Absatz
\end{verbatim}

\subsubsection{Die fortlaufende Liste}

Normalerweise fängt jede neue Liste wieder bei 1 an. Es gibt jedoch eine
besondere Form, die fortlaufend über das gesamte Dokument läuft,
vergleichbar mit den caption-Zählern für \texttt{figure} oder
\texttt{table} Umgebungen.

Auf diese Zähler kann später Bezug genommen werden.

\begin{verbatim}
(@Behauptung) Hier behaupte ich etwas.

Die Behauptung (@Behauptung) wird in (@Beweis) bewiesen.

(@Beweis) Und hier der versprochene Beweis. 
\end{verbatim}

Diese Liste benutzt nicht den
LaTeX~\texttt{\textbackslash{}label}/\texttt{\textbackslash{}ref}-Mechanismus
und erfordert daher keinen zweiten LaTeX-Lauf.

\subsection{Tabellen}

Seit Version 1.8.1.2 werden Tabellen mit dem \texttt{ctable}-Paket
gesetzt. Dadurch hat sich die Qualität der Ausgabe erheblich verbessert.

Mit der Version 1.10 wurde auf \texttt{longtable} umgestellt. Dadurch
sind Tabellen keine Gleitobjetke mehr sondern erscheinen dort wo sie
definiert wurden. Lange Tabellen werden am Seitenende umgebrochen.
Leider landen dadurch auch manchmal Teile von kurzen Tabellen auf der
nächsten Seite.

Die \texttt{longtable} Umgebung ist nicht mit \texttt{beamer}
kompatibel, nach einer Lösung wird noch gesucht.

Es gibt drei Arten von Tabellen. Bei der Eingabe von Tabellen sollte man
auf Tabulatoren verzichten und die Spalten mit Leerzeichen ausrichten.

\begin{verbatim}
 Rechts   Links    Mitte   Standard
-------   ------  -------  --------
     12   12        12           12
    123   123       123         123
    ab    ab        ab          ab

Table: Eine einfache Tabelle
\end{verbatim}

Der Tabellenkopf und die einzelnen Tabellenzeilen müssen in eine Zeile
passen. Die Ausrichtung der Spalten hängt von der Unterstreichung der
Kopfzeile ab.

\begin{itemize}
\item
  Ist die Unterstreichung rechtsbündig und steht auf der linken Seite
  hervor, ist die Ausrichtung rechtsbündig.
\item
  Ist die Unterstreichung linksbündig und steht auf der rechten Seite
  hervor, ist die Ausrichtung linksbündig.
\item
  Steht die Unterstreichung auf beiden Seite hervor, ist die Ausrichtung
  zentriert.
\item
  Ist die Unterstreichung genauso lang wie der Text, wird die
  Standardeinstellung (linksbündig) benutzt.
\end{itemize}

Eine Tabelle muss mit einer Leerzeile abgeschlossen werden. Optional
darf ein Tabellentitel vergeben werden, diese wird durch das
Schlüsselwort \texttt{Table:} (oder nur \texttt{:}) eingeleitet. Wird
ein Titel verwendet, so wird die Tabelle als Gleitobjekt in einer
\texttt{table}-Umgebung formatiert, ansonsten als Tabelle im Fließtext.

Mehrzeilige Tabellen müssen durch jeweils eine Reihe \texttt{-}-Zeichen
eingeschlossen werden. Die Tabellenzeilen werden durch Leerzeilen
getrennt.

\begin{verbatim}
------------------------------------------------------------------
 Zentrierter    Standard    
 Kopf           Kopf       Rechtsbündig  Linksbündig
-------------   --------  -------------  ---------------------
Erste           Zeile             12.0   Beispiel einer 
                                         mehrzeiligen Zeile

Zweite          Zeile              5.0   Noch ein mehrzeilige
                                         Zeile, durch eine
                                         Leerzeile abgetrennt
-------------------------------------------------------------------
\end{verbatim}

Beim dritten Typ handelt es sich um sogenannte Rastertabellen. Der
äußere Rahmen wird natürlich bei LaTeX-Ausgabe nicht gesetzt. In den
Tabellenzellen können z. B. enthalten sein.

\begin{verbatim}
+-----------------------+--------------+-------------------------+
| Frucht                | Preis        | Vorteile                |
+=======================+==============+=========================+
| Banane                | $1.34        | - eingebaute Verpackung |
|                       |              | - leuchtende Farben     |
+-----------------------+--------------+-------------------------+
| Orange                | $2.10        | - heilt Scorbut         |
|                       |              | - ist lecker            |
+-----------------------+--------------+-------------------------+
\end{verbatim}

Mit Version 1.10 wurde ein vierter Tabellentyp eingeführt, die
Pipe-Tabelle. (Pipe ist ein englischer Name für das \texttt{\textbar{}}
Zeichen.) Die Ausrichtung der Zellen wird durch den \texttt{:} bestimmt.
Die Kopfzeile darf leer sein, die Zeile mit den \texttt{-} Zeichen
jedoch nicht, da sie die Ausrichtung definiert. Die \texttt{\textbar{}}
Zeichen in der ersten und letzten Spalte sind optional.

\begin{verbatim}
| Rechts | Links | Standard | Zentriert | 
|-------:|:------|----------|:---------:| 
|    12  | 12    | 12       |     12    | 
|   123  | 123   | 123      |    123    |
|     1  | 1     | 1        |     1     |
\end{verbatim}

\subsection{Titelei}

Angaben zu Titel des Dokuments, den Autoren und dem Datum der Erstellung
werden am Anfang der Datei angegeben:

\begin{verbatim}
% Viele Ziele Veröffentlichung
% Axel Kielhorn
% DTK 2011-3
\end{verbatim}

Lange Titel und mehrere Autoren bricht man auf mehrere Zeilen um:

\begin{verbatim}
% Viele Ziele Veröffentlichung\
  (Multi Target Publishing)
% Axel Kielhorn
  Babel Fisch (Übers.)
% DTK 2011-3
\end{verbatim}

Der \texttt{\textbackslash{}} am Ende der ersten Zeile erzeugt ein
\texttt{\textbackslash{}\textbackslash{}} in der LaTeX-Ausgabe. Leider
führt das bei einigen eBook Lesegeräten zu Problemen, hier sollte man
auf das \texttt{\textbackslash{}} verzichten.

Das ePUB Format fordert ein echtes Datum (2012-02-12) als Argument.

\begin{verbatim}
% Viele Ziele Veröffentlichung
% Axel Kielhorn
% 2012-02-12
\end{verbatim}

\subsection{Fußnoten}

Fußnoten bestehen aus zwei Teilen, der Fußnotenmarkierung und dem
Fußnotentext.

\begin{verbatim}
Dies ist eine Fußnotenmarkierung[^1] und dies eine 
weitere[^fussnote]

[^1]: Hier ist der Fußnotentext

[^fussnote]: Diese Fußnote ist etwas länger.

    Sie enthält einen zweiten Absatz.
\end{verbatim}

\subsection{Hervorhebungen}

Kursiver Text wird durch \emph{ein} \texttt{*} oder \texttt{\_}
eingeschlossen.

Fetter Text wird durch \emph{zwei} \texttt{*} oder \texttt{\_}
eingeschlossen.

Sollen nur Wortteile hervorgehoben werden, ist zwingend der \texttt{*}
zu nutzen, da in Variablennamen häufig \texttt{\_} als Bestandteil des
Namens vorkommen.

\begin{verbatim}
Dieser Text wurde _mit dem Unterstrich hervorgehoben_
und dieser *mit dem Sternchen*.

Für __fetten Text__ benutzt man **zwei** Zeichen.

Variablen_namen_ können Unterstriche enthalten, daher benutzt man 
zum hervor*heben* das Sternchen.
\end{verbatim}

Zum Hochstellen benutzt man das bekannte \texttt{\^{}}, zum Tiefstellen
muss auf die \texttt{\textasciitilde{}} ausgewichen werden, da der
Unterstrich ja schon zum auszeichnen verwendet wird. Die Befehle
schließen das Argument ein.

\begin{verbatim}
H~2~O ist Wasser, 2^10^ ist 1024.
2^2^^2^ ist 2^22^.
\end{verbatim}

Die letzte Zeile mag zuerst verwundern, allerdings handelt es sich hier
nicht um Mathematiksatz sondern um Textexponenten und -indices.

\subsection{Mathematik}

Inline-Mathematik wird wie aus LaTeX~bekannt in \texttt{\$}
einbeschlossen. Der Inhalt wird direkt an LaTeX~weitergeben, daher ist
alles erlaubt, was in LaTeX~möglich ist.

\begin{verbatim}
$2^{2^2} != 2^{22}$
\end{verbatim}

Bei anderen Ausgabeformaten hängt es von jeweiligen Format und den
Optionen beim Aufruf von pandoc ab, was als Ausgabe erzeugt wird.

Abgesetzte Formeln können als rohes LaTeX~eingegeben werden.

\subsection{Rohes LaTeX}

Alles was zwischen einem
\texttt{\textbackslash{}begin}/\texttt{\textbackslash{}end} Paar steht
wird direkt an LaTeX~(oder ConTeXt) weitergegeben und in allen anderen
Formaten ignoriert.

\subsection{Rohes HTML}

Da Markdown ursprünglich zum Erstellen von HTML-Seiten entwickelt wurde,
bietet es die Möglichkeit HTML direkt einzugeben. Bei nicht HMTL
basierten Ausgabeformaten wird es ignoriert.

\subsection{Links}

Bei einem Format, das ursprünglich zum Erstellen von Webseiten gedacht
war, sollte man annehmen, das es mit Hyperlinks umgehen kann. Und so ist
es natürlich auch. Alles, was in einer Zeile zwischen spitzen Klammern
steht, ist ein Link.

\begin{verbatim}
<http://johnmacfarlane.net/pandoc/>
\end{verbatim}

Ein Link kann natürlich auch im Text auftauchen:

\begin{verbatim}
Die Dokumentation zu pandoc befindet sich auf der
[pandoc Webseite](http://johnmacfarlane.net/pandoc/).
\end{verbatim}

\subsection{Bilder}

Bilder sind eine besondere Form des Links. Wenn vor dem Link ein
\texttt{!} steht, wird dieser als Link auf ein Bild interpretiert.

\begin{verbatim}
![Ein Blaues Bild](blau.jpg "Blaues Bild")
\end{verbatim}

Das Bild wird in einer \texttt{figure}-Umgebung gesetzt, der Text in den
eckigen Klammern wird als Bildunterschrift verwendet.

Möchte man das Bild im Fließtext einbinden, darf der Link nicht allein
in einer Zeile stehen.

\begin{verbatim}
Das ![Rotes Quadrat](rot.png "rote Quadrat") im Fließtext.
\end{verbatim}

Es gibt keine Möglichkeit, die Bilder zu skalieren, sie müssen in der
richtigen Größe und Auflösung vorliegen.

\section{Am Wegesrand}

Pandoc kann auch ConTeXt-Dateien erstellen. Damit könnte man vorhandene
LaTeX-Dateien konvertieren, um den Einstieg in ConTeXt~zu finden.

ConTeXt~kann mithilfe des Filter Moduls sogar direkt Markdown bearbeiten
indem es pandoc als Filter aufruft. Näheres hierzu gibt es im Pandoc
Extra Wiki~\autocite{pandoc-extra}.

\section{Reiseliteratur: Große Dokumente \ldots{}}

Standardmäßig erstellt pandoc Dokumente ohne Abschnittsnummerierung und
ohne Inhaltsverzeichnis. Bei kurzen Texten ist das sicher kein Problem,
bei längere Texten muss man nicht darauf verzichten. Mit:

\begin{verbatim}
pandoc -f markdown -t latex --number-sections 
    -o quelle.tex quelle.md
\end{verbatim}

erhält man die Abschnittsnummern und mit:

\begin{verbatim}
pandoc -f markdown -t latex --toc -o quelle.tex quelle.md
\end{verbatim}

das Inhaltsverzeichnis. Beides kann man natürlich auch kombinieren.

Die Tiefe des Inhaltsverzeichnisses kann man ab Version 1.10
beeinflussen:

\begin{verbatim}
 pandoc -f markdown -t latex --toc-depth=3 -o quelle.tex quelle.md
\end{verbatim}

Die Option \texttt{-{}-toc} wird dabei implizit gesetzt, muss also nicht
angegeben werden. Das funktioniert auch bei HTML und epub.

\section{\ldots{}~oder ganze Bücher}

Wenn keine Dokumentklasse angegeben wird, benutzt pandoc standardmäßig
die Klasse \texttt{article}, bzw. wenn die Option \texttt{-{}-chapters}
gewählt wurde, die Klasse \texttt{book}.

Durch das setzen einer Variablen, kann man die gewünschte Dokumentklasse
auswählen, dabei werden \texttt{report}, \texttt{book}, \texttt{memoir},
\texttt{scrreprt} und \texttt{scrbook} automatisch erkannt und die
Option \texttt{-{}-chapters} gesetzt.

Bei älteren pandoc Versionen (\textless{}1.9) war dies nicht möglich.
Pandoc hat hier die Dokumentklasse aus der Template Datei gelesen und
bei \texttt{report}, \texttt{book} und \texttt{memoir} auf Kapitel als
oberste Gliederungsebene umgeschaltet.

Bei \texttt{scrreprt} und \texttt{scrbook} musste man Kapitel durch eine
Option aktivieren:

\begin{verbatim}
# pandoc >=1.9
pandoc -f markdown -t latex -V documentclass=srcreprt -s
    -o quelle.tex quelle.md

# pandoc <1.9
pandoc -f markdown -t latex --template=./komascript.latex 
    --chapters -o quelle.tex quelle.md
\end{verbatim}

Die Datei \texttt{monster.latex} ist ein Versuch möglichst viele
konfigurierbare Optionen in eine Vorlage zu bringen. Durch die Vielzahl
an Optionen ist diese Datei nur noch über Skripte sinnvoll nutzbar, da
kaum jemand alle Optionen jedes mal von Hand eintippen möchte.

Der Quelltext kann in mehrere Dateien aufgeteilt werden, allerdings
bietet pandoc keinen mit
\texttt{\textbackslash{}include}/\texttt{\textbackslash{}includeonly}
vergleichbaren Mechanismus. Stattdessen hängt man die Dateinamen einfach
an den Befehlsaufruf an:

\begin{verbatim}
pandoc -f markdown -t latex --number-sections --toc 
    --template=./report.latex -o md-test.tex 
    md-test-intro.md 
    md-test-ch1.md 
    md-test-ch2.md 
    md-test-ch3.md
\end{verbatim}

\section{Kein Ziel ohne Quellen}

Verweise ins Literaturverzeichnis werden in eckige Klammern gesetzt und
erhalten das \texttt{@}-Zeichen sowie den Zitierschlüssel. Es können
mehrere Quelle in einem Befehl zitiert werden. Kann ein Verweis nicht
aufgelöst werden, so bleibt der Zitierschlüssel im Text, es gibt keine
Fehlermeldung!

\begin{verbatim}
<http://johnmacfarlane.net/pandoc/>[@pandoc]
Dieser Artikel erschien in [@Ziele-dtk; @Ziele-TUGboat; @Ziele-ct]
\end{verbatim}

Pandoc kann \texttt{bibTeX}-Dateien lesen und mit Hilfe von CSL
(Citation Style Language) formatieren. Stile für verschiedene Zeitungen
und Organisationen stehen unter~\autocite{csl} zur Verfügung. Ohne
explizite Angabe wird ein Chicago Autor--Datum Stil verwendet.

Das Literaturverzeichnis wird an das Dokument angehängt, dabei wird die
letzte Überschrift als Titel des Verzeichnisses benutzt. Sollte es nach
dieser Überschrift noch weiteren Text geben, so wird er vor dem
eigentlichen Verzeichnis ausgegeben, z. B. für eine Einleitung zum
Literaturverzeichnis.

\begin{verbatim}
pandoc -f markdown -t latex --bibliography quelle.bib --csl apa.csl 
    -o quelle.tex quelle.md
\end{verbatim}

Diese Methode hat den Vorteil, das sie mit allen Ausgabeformaten
funktioniert und keinen weiteren TeX-Lauf benötigt. Sie hat jedoch den
Nachteil, das im erzeugten PDF keine Links auf das Literaturverzeichnis
verweisen. Außerdem gibt es Probleme mit LaTeX-Befehlen in der
\texttt{bib}-Datei, das TeX~in TeXnische Komödie wird falsch
wiedergegeben.

Bei der Ausgabe einer TeX-Datei kann man alternativ auch
\texttt{biblatex} verwenden. In Verbindung mit \texttt{hyperref}
entstehen so Links in das Literaturverzeichnis. Die Sortierung erfolgt
dann mit \texttt{bibtex} oder \texttt{biber}, die Formatierung über
entsprechende \texttt{biblatex} Stile. Es sind zusätzliche LaTeX~Läufe
erforderlich.

\begin{verbatim}
pandoc -f markdown -t latex --bibliography quelle.bib --biblatex 
    -o quelle.tex quelle.md
\end{verbatim}

\section{Fazit}

Anders als bei LaTeX ist die Einarbeitung in Markdown sehr einfach.
Solange die Standardmöglichkeiten ausreichen bietet Markdown als
Eingabeformat nur Vorteile: weniger Tipparbeit, übersichtlicherer Text.
Trotzdem befindet sich im Hintergrund ein komplettes LaTeX System, auf
das jederzeit zurückgreifen kann. Zum Anpassen der Template Datei sind
LaTeX Kenntnisse erforderlich und man sollte auch im Betrieb einen LaTeX
Experten zur Hand haben, falls es mal hakt. Aber es hakt deutlich
weniger, jedenfalls solange man auf rohes LaTeX verzichtet.

Reichen irgendwann die Fähigkeiten von Pandoc nicht mehr aus, hat man ja
immer noch eine LaTeX Datei als Zwischenprodukt. Diese kann dann z. B.
mit \texttt{\textbackslash{}index} Befehlen erweitert werden.

Die Möglichkeit neben PDF~auch andere Formate zu erstellen ist ein
deutlicher Gewinn. Nun kann aus einer Quelle ein pBuch, ein eBuch und
eine Webseite erstellt werden.

Was fehlt ist ein Editor der Pandoc per Knopfdruck aufruft und somit den
Ausflug in die Kommandozeile erspart. Natürlich ist eine Integration in
Vim oder Emacs kein Problem, aber das sind nicht die Editoren, die ein
Einsteiger nutzen wird. Für TeXshop gibt es bereits engines für die
wichtigsten Aufgaben (Zielformate PDF und ePUB), bei anderen Editoren
sieht es noch schlecht aus.

\section{Lizenz}

Dieser Artikel und das Begleitmaterial unterliegen eine Creative Commons
Namensnennung--Weitergabe unter gleichen Bedingungen Lizenz der Version
3.0. Weitere Informationen unter
\url{http://creativecommons.org/licenses/by-sa/3.0/de/}

\section{Anhang}

Im Begleitmaterial befinden sich folgende Dateien:

\begin{description}
\item[md-test.md]
Die Beispiele aus diesem Artikel
\item[md-test.bib]
Eine \texttt{bibtex} Datei für das Literaturverzeichnis.
\item[md-test.tex]
Konvertiert nach LaTeX.
\item[md-test.pdf]
Mit pdfLaTeX~gesetzt.
\item[md-test.epub]
Konvertiert nach ePUB
\item[default-de.latex, report-de.latex, komascript.latex,
monster.latex]
Ein paar LaTeX-Vorlagen zum Erstellen deutscher Texte.
\item[reference-de.odt]
Eine OpenOffice-Vorlage zum Erstellen deutscher Texte.
\item[dtk2mdtex.sed]
Eine sed Datei zum Entfernen der \texttt{dtk}-spezifischen Auszeichnung.
\item[pandoc.pdf]
Die pandoc-Manpage.
\item[pandoc-markdown.pdf]
Die Manpage, die den Markdown Syntax von pandoc beschreibt.
\item[Kommandozeilen.txt]
Eine Sammlung von Kommandozeilen, um Tipparbeit zu sparen.
\end{description}

\printbibliography

\end{document}
